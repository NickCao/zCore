\documentclass{ctexbeamer}

\usetheme{Berkeley}

\title{Distributed Operating System}
\subtitle{Filesystem actually}

\author{曹隽诚 \and 李晋}
\date{2022年10月13日}

\begin{document}
\frame{\titlepage}

\section{Introduction}

\begin{frame}
\frametitle{Distributed Operating System}
\framesubtitle{Plan 9 from Bell Labs}
\begin{quotation}
The view of the system is built upon three principles. First, resources are named and accessed like files in a hierarchical file system. Second, there is a standard protocol, called 9P, for accessing these resources. Third, the disjoint hierarchies provided by different services are joined together into a single private hierarchical file name space.
\end{quotation}
\end{frame}

\begin{frame}
\frametitle{Distributed Operating System}
\framesubtitle{Transparency}
\begin{itemize}
  \item files (files are just, files)
  \item inter-process communication (unix domain socket)
  \item process management (procfs)
\end{itemize}
\begin{block}{Takeaway}
  Not everything is a file, everything is accessed as a file

  A distributed operating system is just a distributed filesystem
\end{block}
\end{frame}

\section{Design}
\begin{frame}
\frametitle{Distributed Filesystem}
\framesubtitle{Interface}
\begin{block}{VFS (virtual file system)}
  \begin{itemize}
    \item allow processes to access local, remote or any filesystem transparently over an consistent \emph{interface}
    \item VFS in Linux takes an inode based design
  \end{itemize}
\end{block}
\begin{block}{inode}
  \begin{itemize}
    \item an inode can be a regular file, a directory, a FIFO, a device or any other beasts
    \item within a filesystem, every inode has a \emph{unique} inode number
    \item inode numbers are allocated by the filesystem, and has no meaning other than an opaque identifier to the processes
  \end{itemize}
\end{block}
\end{frame}

\begin{frame}
\frametitle{Distributed Filesystem}
\framesubtitle{Identifier allocation}
\begin{verse}
  All distrubuted systems, however designed, need means for syncronization, just some of them need less, by keeping resources local.
\end{verse}
\begin{itemize}
  \item all nodes are allocated an unique node number on creation
  \item inode numbers are allocated locally, as a combination of node number and a sequence number
  \item local inodes are accessed as-is
  \item remote inodes are self-descriptive, the target node can be located without coordination
\end{itemize}
\end{frame}

\begin{frame}
\frametitle{Distributed Filesystem}
\framesubtitle{File handle}
\begin{quotation}
  There are 2 hard problems in computer science: cache invalidation, naming things, and off-by-1 errors.
  \flushright - Leon Bambrick
\end{quotation}
Inode themselves, are local states, but file handles, are not. Just google ``nfs stale file handle'', and you will get 137,000 results. File handles, are just caches of the presence of specific inodes.

But there is a simple solution: like NFS, we just fail.
\end{frame}

\begin{frame}
\frametitle{Distributed Filesystem}
\framesubtitle{Remote procedure call}
The whole VFS interface, served over the network
\begin{block}{Message}
  We use protobuf, a language-neutral, platform-neutral, extensible mechanism for serializing structured data, to encode messages
\end{block}
\begin{block}{Exchange}
  Messages are exchanged over TCP as a sequence of TLV, for simplicity of implementation, in a request and response manner
\end{block}
\end{frame}
  
\end{document}
